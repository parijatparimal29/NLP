\documentclass{article}

%Page format
\usepackage{pdfpages}
\usepackage{fancyhdr}
\usepackage[margin=1in]{geometry}

%Math packages and custom commands
\usepackage{framed}
\usepackage{tikz}
\usepackage[utf8]{inputenc}
\usepackage[margin=1in]{geometry}
\usepackage{mathtools,amsthm,bbm}
\usepackage{enumitem,amssymb}
\newtheoremstyle{case}{}{}{}{}{}{:}{ }{}
\theoremstyle{case}
\newtheorem{case}{Case}
\DeclareMathOperator{\R}{\mathbb{R}}
\DeclareMathOperator{\E}{\mathbb{E}}
\DeclareMathOperator{\Var}{\text{Var}}
\DeclareMathOperator{\Cov}{\text{Cov}}
\newcommand{\bvec}[1]{\mathbf{#1}}
\renewcommand{\P}{\mathbb{P}}
\newcommand{\norm}[2][2]{\| #2\|_{#1}}
\newcommand\eqdef{\stackrel{\rm def}{=}} % Equal by definition

\definecolor{shadecolor}{gray}{0.9}

\theoremstyle{definition}
\newtheorem*{answer}{Answer}
\newcommand{\note}[1]{\noindent{[\textbf{NOTE:} #1]}}
\newcommand{\hint}[1]{\noindent{[\textbf{HINT:} #1]}}
\newcommand{\recall}[1]{\noindent{[\textbf{RECALL:} #1]}}

\newlist{todolist}{itemize}{2}
\setlist[todolist]{label=$\square$}
\usepackage{pifont}
\newcommand{\cmark}{\ding{51}}%
\newcommand{\xmark}{\ding{55}}%
\newcommand{\done}{\rlap{$\square$}{\raisebox{2pt}{\large\hspace{1pt}\cmark}}%
\hspace{-2.5pt}}
\newcommand{\wontfix}{\rlap{$\square$}{\large\hspace{1pt}\xmark}}


\title{\textbf{CSCI-GA 2590: Natural Language Processing} \\Representing and Classifying Text}
\author{Name \\
NYU ID}
\date{}

\lhead{NYU ID}
\chead{Representing and Classifying Text}
\rhead{\today}
\lfoot{}
\cfoot{CSCI-GA 2590: Natural Language Processing --- Fall 2020}
\rfoot{\thepage}
\renewcommand{\headrulewidth}{0.4pt}
\renewcommand{\footrulewidth}{0.4pt}
\pagestyle{fancy}
\setlength{\parindent}{0pt}

\begin{document}

\maketitle

\textbf{Collaborators:} \\
\textit{By turning in this assignment, I agree by the honor code of the College of Arts and Science at New York University and declare
that all of this is my own work.} \\

Welcome to your first assignment! The goal of this assignment is to get familiar with basic text classification models through both mathematical analysis and hands-on feature engineering.
\textbf{Before you get started, please read the Submission section thoroughly}.

\section*{Submission}
Submission is done on Gradescope. \\

\textbf{Written:} When submitting the written parts, make sure to select \textbf{all} the pages that contain part of your answer for that problem, or else you will not get credit.
You can either directly type your solution between the \texttt{shaded} environments in the released \texttt{.tex} file,
or write your solution using pen or stylus. 
A \texttt{.pdf} file must be submitted.

\textbf{Programming:} Questions marked with ``coding'' next to the assigned to the points require a coding part in \texttt{submission.py}.
Submit \texttt{submission.py} and we will run an autograder on Gradescope. You can use functions in \texttt{util.py}. However, please do not import additional libraries (e.g. \texttt{numpy}, \texttt{sklearn}) that aren't mentioned in the assignment, otherwise the grader may crash and no credit will be given.
You can run \texttt{test.py} to test your code but you don't need to submit it. Please do not change the name of the functions as it might cause the grader to crash. Also after running the autograder, the graders will manually check the code. No credits will be awarded for hard coding the answers


\section*{Problem 1: Naive Bayes classifier}
In this problem, we will study the \emph{decision boundary} of multinomial Naive Bayes model for binary text classification.
The decision boundary is often specificed as the level set of a function:
$\{x\in\mathcal{X} : h(x) = 0\}$,
where $x$ for which $h(x) > 0$ is in the positive class and
$x$ for which $h(x) < 0$ is in the negative class.

\begin{enumerate}
    \item {[2 points]} Give an expression of $h(x)$ for the Naive Bayes model $p_\theta(y\mid x)$, where $\theta$ denotes parameters of the model.
    

    \newpage
\item {[3 points]} Recall that for multinomial Naive Bayes,
        we have the input $X=(X_1, \ldots, X_n)$ where $n$ is the number of words in an example.
        In general, $n$ changes with each example but we can ignore that for now.
        We assume that $X_i\mid Y=y \sim \text{Categorical}(\theta_{w_1,y}, \ldots \theta_{w_m,y})$ where $Y\in \{0, 1\}$, $w_i\in\mathcal{V}$, and $m=|\mathcal{V}|$ is the vocabulary size.
        Further, $Y\sim\text{Bernoulli}(\theta_1)$.
        Show that the multinomial Naive Bayes model has a linear decision boundary,
        i.e. show that $h(x)$ can be written in the form $w\cdot x + b=0$.
        \recall{
            The categorical distribution is a multinomial distribution with one trial.
            Its PMF is
            $$
            p(x_1,\ldots, x_m) = \prod_{i=1}^m\theta_{i}^{x_i} \;,
            $$
            where $x_i = \mathbbm{1}[x=i]$, $\sum_{i=1}^m x_i = 1$,
            and $\sum_{i=1}^m \theta_i = 1$.
        }
 
    \newpage
\item {[2 points]} In the above model, $X_i$ represents a single word, i.e. it's a unigram model.
    Think of an example in text classification where the Naive Bayes assumption might be violated.
    How would you allieviate the problem?


\item {[2 points]} Since the decision boundary is linear, the Naive Bayes model works well if the data is linearly separable.
        Discuss ways to make text data more separable and its influence on model generalization.


\newpage
\end{enumerate}

\section*{Problem 2: Skip-gram model}
In this problem, we will gain some insights into the skip-gram model.
In particular, we will show that it
encourages the inner product between two word embeddings
to be equal to their pointwise mutual information (PMI) (up to a constant).

Recall that to evaluate the objective function of the skip-gram model
we need to enumerate over the entire vocabulary $\mathcal{V}$,
which can be quite expensive in practice.
Note that the reason we need to enumrate the vocabulary
is to obtain a probability distribution of neighboring words.
Let's define the \emph{neighborhood} of a word $w$ to be a size-$L$ window around it
and words within the window are $w$'s neighbors.
Instead of modeling the distribution of words,
we can model the distribution of an indicator: whether two words are neighbors.
Let $Y$ be the indicator random variable.
We model it by a Bernoulli distribution
$$
p_\theta(y=1\mid w, c) = \frac{1}{1 + e^{-u_c\cdot v_w}} \;,
$$
where $c \in \mathcal{V}$ is within a window centered at $w\in \mathcal{V}$,
$u_c, v_w$ represent embeddings of $c$ and $w$,
and $\theta$ is the model parameter, i.e. word embedding $v$'s and context embedding $u$'s.
What about the negative observations ($y=0$)?
A naive way is to consider all words that do not occur in the neighborhood of $w$, however, this is as expensive as our original objective.
One solution is \textbf{negative sampling}, where we only consider a small number of non-neighboring words.

Let $D=\{(w, c)\}$ be the set of all word-neighbor pairs observed in the training set.
For each pair $(w,c)$, we generate $k$ negative neighbors $c_N$ by sampling from a distribution $p_N(\cdot)$ over the vocabulary.

\begin{enumerate}
    \item {[3 points]} Write down a learning objective that maximizes the log-likelihood of $D$
        and the \emph{expected} log-likelihood of the negative examples.
        Note that $C_N$ is the random variable in the expectation.
        You can use the Bernoulli model $p_\theta(y\mid w, c)$ in the expression.


    \newpage
\item {[2 points]} Let 
        $$
        \sigma(\alpha) = \frac{1}{1+e^{-\alpha}} \;.
        $$
        Show that
        $$
        \frac{d}{d\alpha}\log \sigma(\alpha) = \frac{1}{1+e^\alpha} \;.
        $$

    
    \newpage
\item {[3 points]} Let's consider one pair $(w=b,c=a)$.
        Let $\alpha = u_a\cdot v_b$.  
        Show that the optimal solution of $\alpha$ is
        $$
        \alpha^\ast = \log \frac{A}{B} \;,
        $$
        where
        \begin{align} 
            A &= \sum_{(w,c)\in D}\mathbbm{1}[w=b,c=a] \\
            B &= k\sum_{(b,\cdot)\in D}p_N(a) \;.
        \end{align} 
        \hint{
            Let $\ell(\theta)$ be the objective.
            Solve for $\alpha$ in $\frac{\partial \ell}{\partial \alpha}=0$.
            You can use results from previous questions.
        }
    \newpage
\item {[2 points]} Suppose the distribution of negative words $p_N(w)$ for $w\in\mathcal{V}$ is a categorical distribution given by
        $$
        p_N(w) = \frac{\text{count}(w, \cdot, D)}{|D|} \;,
        $$
        where $\text{count}(w, \cdot, D)  = \sum_{c\in\mathcal{V}} \text{count}(w, c, D)$.
        Note that $p_N(w)$ is simply the fraction of unigram $w$ in $D$. 
        Show that
        $$
        \alpha^\ast = \text{PMI}(b,a) - \log k  \;,
        $$
        where the PMI score of word co-occurence in $D$ is defined as
        $$
        \text{PMI}(b,a) \eqdef \log
        \frac{\text{count}(b,a,D)|D|}
        {\text{count}(b,\cdot, D) \text{count}(a,\cdot, D)}
        \;.
        $$
    \newpage
\item {[2 points]} Let's denote the unigram model defined above by $p_{\text{unigram}}(w)$.
        In practice, we often prefer to use $p_N(w) \sim p_{\text{unigram}}^\beta(w)$
        where $\beta\in [0, +\infty]$.
        Describe the desired range of $\beta$ and explain why. 

\end{enumerate}

\newpage
\section*{Problem 3: Natural language inference}
In this problem, you will build a logistic regression model for textual entailment.
Given a \textit{premise} sentence,
and a \textit{hypothesis} sentence,
we would like to predict whether the hypothesis is \textit{entailed} by the premise,
i.e. if the premise is true, then the hypothesis must be true.

Example:
\begin{center}
    \begin{tabular}{lll}
        label & premise & hypothesis \\
        \hline
        entailment & The kids are playing in the park & The kids are playing \\
        non-entailment &The kids are playing in the park & The kids are happy 
    \end{tabular}
\end{center}

\begin{enumerate}
    \item {[1 point]} Given a dataset $D=\{(x^{(i)}, y^{(i)})\}_{i=1}^n$ where $y\in\{0,1\}$, let $\phi$ be the feature extractor and $w$ be the weight vector. Write the maximum log-likelihood objective as a \textit{minimization} problem.

    \newpage
    \item {[3 point, coding]} We first need to decide the features to represent $x$.
        Implement \texttt{extract\_unigram\_features} which returns a BoW feature vector for the premise and the hypothesis.


    \newpage
    \item {[2 point]} Let $\ell(w)$ be the objective you obtained above.
    Compute the gradient of $\ell_i(w)$ given a single example $(x^{(i)}, y^{(i)})$.
        Note that $\ell(w) = \sum_{i=1}^n \ell_i(w)$.
        You can use $f_w(x) = \frac{1}{1 + e^{-w\cdot\phi(x)}}$
        to simplify the expression.

    \newpage
    \item {[5 points, coding]} Use the gradient you derived above to implement \texttt{learn\_predictor}.
        You must obtain an error rate less than 0.3 on the training set and less than 0.4 on the test set to get full credit \emph{using the unigram feature extractor}.


    \newpage
    \item {[3 points]} Discuss what are the potential problems with the unigram feature extractor
        and describe your design of a better feature extractor. 


    \newpage
    \item {[3 points, coding]}
        Implement your feature extractor in \texttt{extract\_custom\_features}.
        You must get a lower error rate on the dev set than what you got using the unigram feature extractor to get full credits.


    \newpage
\item {[3 points]} When you run the tests in \texttt{test.py}, it will output a file \texttt{error\_analysis.txt}. (You may want to take a look at the \texttt{error\_analysis} function in \texttt{util.py}).
    Select five misclassified examples.
        For each example, briefly state your intuition for why the classifier got it wrong, and what information about the example will be needed to get it right.


    \newpage
\item {[3 points]} Change \texttt{extract\_unigram\_features} such that it only extracts features from the hypothesis.
    How does it affect the accuracy? Is this what you expected? If yes, explain why. If not, give some hypothesis for why this is happening.
    Don't forget to change the function back before your submit. You do not need to submit your code for this part.

\end{enumerate}

\end{document}